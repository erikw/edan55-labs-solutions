\documentclass{tufte-handout}
\usepackage{amsmath,amsthm}

\usepackage{booktabs}
\usepackage{graphicx}
\usepackage{tikz}

\newtheorem{claim}{Claim}[section]
\title{Collatz 81}
\author{}

\begin{document}
\maketitle


\section{Collatz 81 Report}


by Erik Westrup \& Dimitry Basavin.

\subsection{1. Dictionaries}

We let $C_N$ denote the (``flattened'') sequence of the sequences
$c_1,\ldots, c_N$. For instance, $C_3$ is 1, 2, 1, 3, 10, 5, 16, 8, 4,
2, 1.


The following table gives the maximum value appearing in $C_N$, the number
of distinct values in $C_N$ (i.e., the cardinality of $C_N$ viewed as
a set), and the total length of the sequence $|c_1|+\cdots+|c_N|$,
for increasing values of $N$.

\medskip
\begin{tabular}{rrrr}
  \toprule
  $N$ & $\max C_N$ & $|C_N|$ & len($C_N$) \\
  \midrule
  10 & 52 & 22 & 77  \\
  100 &  9232 & 251 & 3242 \\
  1,000 & 250504 & 2228 & 60542 \\
  10,000 & 27114424 & 21664 & 859666 \\
  100,000 & 1570824736 & 217212 & 10853840\\
  1,000,000 & 56991483520 & 2168611 & 132434424 \\
  \bottomrule
\end{tabular}


\subsection{2. Quadratic Time}

The first solution in small space uses the following idea: For given
$N$, produce every value of $C_N$ (without storing the entire
sequence!)
to determine $\max C_N$.
Start a counter at 0.
Then, for every $i=1,\ldots, \max C_N$, produce the entire sequence to
see if $i$ appears.
If so, increase the counter.
The running time will grow as the product of $\operatorname{len}C_N$ and $ \max C_n$.

The largest $N$ for which this idea works within 60 seconds on our
machine was $100$.


\subsection{3. Randomized Approximation in Small Space}

The following table shows the output of our implementation of
Algorithm D, together with the error (in percent) relative to the
correct values of $|C_N|$ computed in the first part of the report.

\medskip
\begin{tabular}{rrr}
  \toprule
  $N$ &   output & relative error \\
  \midrule
  10 & 21 & 0.045 \\
  100 & 250 & 0.004 \\
  1,000 & 2242 & 0.0063 \\
  10,000 & 21510 & 0.0071 \\
  100,000 & 222251 & 0.023 \\
  1,000,000 & 2539469 & 0.17 \\
  \bottomrule
\end{tabular}
\end{document}
 
