\documentclass{tufte-handout}
\usepackage{amsmath,amsthm}

\usepackage{booktabs}
\usepackage{graphicx}
\usepackage[separate-uncertainty]{siunitx}
\usepackage{tikz}

\newtheorem{claim}{Claim}[section]
\title{\sf Lab Report: Marking Trees}
\date{}
\begin{document}
\maketitle

by Erik Westrup. 

\subsection{Results}

For $i\in\{1,2,3\}$, the number of rounds $R_i$ spent until the tree
is completely marked in process $i$ is given in the following table.
The table shows the result of $[\cdots]$ repeated
trails.

\input{stats_table.tex}

\subsection{Analysis}

Our experimental data indicates that $\mathbf E [R_1]$ is [$\ldots$],
while $\mathbf E[R_2]$ [$\cdots$], and $\mathbf E[R_3]$
[$\cdots$].

Theoretically, the behaviour of $R_1$ can be explained as follows: [$\cdots$] 


\end{document}
