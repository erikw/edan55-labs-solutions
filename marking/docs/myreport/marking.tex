\documentclass{tufte-handout}
\usepackage{amsmath,amsthm}

\usepackage{booktabs}
\usepackage{graphicx}
\usepackage[separate-uncertainty]{siunitx}
\usepackage{tikz}

\newtheorem{claim}{Claim}[section]
\title{\sf Lab Report: Marking Trees}
\date{}
\begin{document}
\maketitle

by Erik Westrup. 

\subsection{Results}

For $i\in\{1,2,3\}$, the number of rounds $R_i$ spent until the tree
is completely marked in process $i$ is given in the following table.
The table shows the result of $[\cdots]$ repeated
trails.

\medskip\noindent
\begin{tabular}{
    S[table-format = 7]
    S[table-format = 1.1(1)e1]
    S[table-format = 1.1(1)e1]
    S[table-format = 1.1(1)e1]
  } 
% WARNING: This table the (brilliant) siunitx package.
% This allows typesetting of nicely aligned numbers.
% If this is too much to absorb, just use a normal Latex table.
% (Or do the table in another tool, export as PDF, and include it.)
% Or do the whole report in your favourite word processor instead.
\toprule
{ $N$ } & { $R_1$ } & {$R_2$} & {$R_3$} \\\midrule
3 & 2.5\pm 0.9 \\
7 & \\
15 & \\
31 & \\
63 & \\
127 & \\
255 & \\
511 & \\
1023 & 3.3 \pm 0.8 e4\\
$\vdots$ \\
524287 & 3.2 \pm 0.2 e6 \\
1048575 \\\bottomrule
\end{tabular}

\subsection{Analysis}

Our experimental data indicates that $\mathbf E [R_1]$ is [$\ldots$],
while $\mathbf E[R_2]$ [$\cdots$], and $\mathbf E[R_3]$
[$\cdots$].

Theoretically, the behaviour of $R_1$ can be explained as follows: [$\cdots$] 


\end{document}
