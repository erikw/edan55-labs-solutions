\documentclass{tufte-handout}
\usepackage{amsmath,amsthm}

\usepackage{booktabs}
\usepackage{graphicx}
%\usepackage[separate-uncertainty]{siunitx}
\usepackage{tikz}

\newtheorem{claim}{Claim}[section]
\title{\sf Lab Report: Marking Trees}
\date{}
\begin{document}
\maketitle

by Erik Westrup. 

\subsection{Results}

For $i\in\{1,2,3\}$, the number of rounds $R_i$ spent until the tree
is completely marked in process $i$ is given in the following table.
The table shows the result of $[10]$ repeated
trails.

\input{data_table.tex}

\section{Analysis}

\subsection{Empirical}
Our experimental data indicates that $\mathbf E [R_1]$ is upper bounded by the expected value of a gemoetric distribution i.e. $O(nlogn)$ and below by $O(n)$.

For R2 is upper bounded by $(n)$ and lower bounded by the best case $(n/2)$

For R3 we have the expected value of $n/2$ eactly and $\Theta(n/2)$.

while $\mathbf E[R_2]$ [$\cdots$], and $\mathbf E[R_3]$
[$\cdots$].

\subsection{Theorideical}
Theoretically, the behaviour of $R_1$ should be lower bounded by $O(n/2)$ too but we didn't see that in our empirical data.

Turns out that for $n/2$ of the nodes, the leafs, this is a coupons collectors problem. Our empirical data followed this as the upper bound of $n(logn)$. The mark cascading propagates biased upwards i.e. it's more likely to mark up wards. The non-leaf nodes will be marked with hihgh probability anyways so focus on the coupons problem on pairs of leaf.

There are $n/4$ pairs of leafs so the bound is $O(\frac{n}{4}log(\frac{n}{4}))$.


\end{document}
