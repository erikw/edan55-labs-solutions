\documentclass{tufte-handout}
\usepackage{amsmath,amsthm}

\usepackage{pgfplots}
\pgfplotsset{width=\textwidth,compat=1.5.1}

\newtheorem{claim}{Claim}[section]
\title{\sf Pagerank}
%\date{\GITAuthorDate}
%\author{Thore Husfeldt}

\begin{document}
\maketitle

\section{Pagerank Lab Report}

by Erik Westrup \& Dmitry Basavin

\subsection{Transition probabilities}

The transition matrix for the graph described in three.txt
is\ldots
\begin{equation*}
P = 
\left(
\begin{array}{cccc}
0.          & 0.61666667  & 0.33333333  & 0.05\\
0.05        & 0.          & 0.9         & 0.05\\
0.33333333  & 0.33333333  & 0.          & 0.33333333\\
0.33333333  & 0.33333333  & 0.33333333  & 0.
\end{array}
\right)\,,
\end{equation*}
and its 10th power is
\begin{equation*}
P^{10} = 
\left(
\begin{array}{cccc}
0.18839185  & 0.29012249  & 0.37320924  & 0.14827642\\
0.18829185  & 0.28999904  & 0.37342054  & 0.14828858\\
0.18840358  & 0.29001076  & 0.37319144  & 0.14839423\\
0.18840358  & 0.29001076  & 0.37319708  & 0.14838858\\
\end{array}
\right)\,.
\end{equation*}

The transition matrix $P$ can be broken down into  $P = \alpha(H + D)
+ \frac{1-\alpha}{n} \bf 1$, where 

\begin{equation*}
H =
\left(
\begin{array}{cccc}
0 & 0.666666666667 & 0.333333333333 & 0 \\
0 & 0 & 1.0 & 0 \\
0.333333333333 & 0.333333333333 & 0 & 0.333333333333 \\
0 & 0 & 0 & 0
\end{array}
\right)\,,
\end{equation*}

\begin{equation*}
D =
\left(
\begin{array}{cccc}
0.00 & 0.00 & 0.00 & 0.00 \\
0.00 & 0.00 & 0.00 & 0.00 \\
0.00 & 0.00 & 0.00 & 0.00 \\
0.25 & 0.25 & 0.25 & 0.25 \\
\end{array}
\right)\,.
\end{equation*}

\subsection{Results}

The following table gives the top hits, i.e., the 5 first vertices of
each graph sorted by page rank, using $\alpha = \frac{85}{100}$.



\medskip
\begin{fullwidth}
\small
\begin{tabular}{lcccccccccc}
three.txt & 2 (36.6\%) & 1 (27.5\%) & 0 (18.4\%) & 3 (17.3\%) \\
three.txt & 2 (36.358721\%)1 (27.954581\%)0 (17.843349\%)3 (17.843349\%)
tiny.txt & [\ldots] &\\
medium.txt &\\
wikipedia.txt & \\
p2p-Gnutella08-mod.txt &
\end{tabular}
\end{fullwidth}

\bigskip The following table gives the number of random walk steps and
(scalar) multiplications needed for each graph until the results were
stable to within 2 decimal places.

\medskip
\begin{fullwidth}
\small
\begin{tabular}{lcccccccccc}
Graph & \# transitions  & \# multiplications \\
three.txt & 54,325 \\
tiny.txt &\\
medium.txt &\\
wikipedia.txt & \\
p2p-Gnutella08-mod.txt
\end{tabular}
\end{fullwidth}

\subsection{Optional}

Build a time machine, fly back to the early 1990s.
Start a search engine company based on this idea.


\section{Perspective}

For more thorough introduction to the mathematics
  behind this model, see David Austin, \emph{How Google Finds Your Needle in
  the Web's Haystack}, American Mathematical Society Feature Column,
  2006.

The original paper is Sergey Brin, Lawrence Page, \emph{The anatomy of a
large-scale hypertextual Web search engine}, which also mentions a
bit about the data structure used for storing web page content.
A different model for establishing web page relevance was established
by Kleinberg around the same time as PageRank.
\end{document}
